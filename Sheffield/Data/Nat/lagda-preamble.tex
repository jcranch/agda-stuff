\documentclass{article}
\usepackage{hyperref}

% The following packages are needed because unicode
% is translated (using the next set of packages) to
% latex commands. You may need more packages if you
% use more unicode characters:

\usepackage{amssymb}
\usepackage{bbm}
\usepackage{MnSymbol}

\usepackage[greek,english]{babel}

% This handles the translation of unicode to latex:

\usepackage{ucs}
\usepackage[utf8x]{inputenc}
\usepackage{autofe}

\usepackage{fullpage}

% Some characters that are not automatically defined
% (you figure out by the latex compilation errors you get),
% and you need to define:

\DeclareUnicodeCharacter{8988}{\ensuremath{\ulcorner}}
\DeclareUnicodeCharacter{8989}{\ensuremath{\urcorner}}
\DeclareUnicodeCharacter{8803}{\ensuremath{\overline{\equiv}}}
\DeclareUnicodeCharacter{8718}{\ensuremath{\blacksquare}}
\DeclareUnicodeCharacter{8760}{\ensuremath{\dotminus}}

% Add more as you need them (shouldn’t happen often).

% Using “\newenvironment” to redefine verbatim to
% be called “code” doesn’t always work properly. 
% You can more reliably use:

\usepackage{fancyvrb}

\DefineVerbatimEnvironment
  {code}{Verbatim}
  {} % Add fancy options here if you like.

% \newcommand{\wikiref}{2}{\href{http://wiki.portal.chalmers.se/agda/pmwiki.php?n=#1}{#2}}
